\section{Entwicklungsumgebung}

Dieser Abschnitt beschreibt die notwendingen Werkzeuge und Technologien, die zum Entwickeln dieser Anwendung eingesetzt wurden.
Dies dient zum reproduzieren der Ergebnisse. 


\subsection{Programmiersprache}

Die Software wird in der Programmiersprache Go geschrieben.
Die verwendete Version kann vom GitHub-Repository in der Datei \texttt{go.mod} eingesehen werden.


\subsection{Software}


\subsubsection{IDE und Tools}

\begin{itemize}
    \item \textbf{Visual Studio Code:} Visual Studio Code ist ein quelloffener Quellcode-Editor von Microsoft. Da es verschiedene Platformen unterstützt und viele Erweiterungen (Extensions) anbietet, ist es ein beliebtes Tool für Entwickler.
    \item \textbf{Git:} Git ist ein Versionskontrollsystem, das die Zusammenarbeit erleichtert. Es ermöglicht die Verfolgung von Änderungen im Quellcode.
    \item \textbf{GitHub:} GitHub ist eine Online-Platform für die Versionsverwaltung von Quellcode. Es bietet Funktionen wie Pull-Requests, Issues und Workflow Actions.
\end{itemize}

Visual Studio Code ist emppfohlen, da der Quellcode die passenden Konfigurationen enthält.


\subsubsection{Libraries und Frameworks}

\begin{itemize}
    \item \textbf{bubbletea:} bubblecharm ist ein TUI-Framework, das auf Go basiert. \cite{bubbletea}
    \item \textbf{bubbles:} bubbles ist eine Bibliothek für bubblecharm, die TUI-Komponenten bereitstellt. \cite{bubbles}
    \item \textbf{teatest:} teatest ist ein Test-Framework für bubblecharm-Anwendungen. \cite{teatest_docs}
    \item \textbf{Libp2p:} Libp2p ist ein P2P-Netzwerk-Protokoll. Es bietet eine einfache und flexiblen Schnittstelle zum Erstellen von Peer-to-Peer-Netzwerken. \cite{libp2p_go_getting_started}
\end{itemize}

Die Version der verwendeten Libraries und Frameworks kann in der Datei \texttt{go.mod} im GitHub-Repository eingesehen werden.


\subsection{Installation und Einrichtung}

Um die Anwendung zu installieren und auszuführen, müssen die folgenden Schritte ausgeführt werden:

\begin{enumerate}
    \item Installiere Go:
    \begin{itemize}
        \item Die Version von Go kann auf https://golang.org/ heruntergeladen werden.
        \item Befolge die Anweisungen auf der Website, um Go auf deinem System zu installieren.
        \item Überprüfe die Installation, indem du den Befehl \texttt{go version} in der Kommandozeile ausführst.
    \end{itemize}

    \item Installiere Git:
    \begin{itemize}
        \item Die Version von Git kann auf https://git-scm.com/ heruntergeladen werden.
        \item Befolge die Anweisungen auf der Website, um Git auf deinem System zu installieren.
        \item Überprüfe die Installation, indem du den Befehl \texttt{git --version} in der Kommandozeile ausführst.
    \end{itemize}

    \item Installiere Visual Studio Code:
    \begin{itemize}
        \item Die Version von Git kann auf https://git-scm.com/ heruntergeladen werden.
        \item Befolge die Anweisungen auf der Website, um Git auf deinem System zu installieren.
        \item Überprüfe die Installation, indem du den Befehl \texttt{git --version} in der Kommandozeile ausführst.
    \end{itemize}

    \item Clone das GitHub-Repository:
    \begin{itemize}
        \item Öffne die Kommandozeile und klone das Repository:
        
        \texttt{git clone https://github.com/TIATIP-24-A-a/MumbleMates.git}.
    \end{itemize}
    
    \item Ausführen der Anwendung:
    \begin{itemize}
        \item Wechsle in das Verzeichnis des geklonten Repositorys und führe den Befehl \texttt{go run .} aus.
    \end{itemize}
\end{enumerate}

\newpage





