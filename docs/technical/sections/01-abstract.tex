\section*{Abstract}

Dieses Dokument beschreibt die Implementierung einer einfachen Peer-to-Peer (P2P) Chat-Anwendung namens MumbleMates, die in Go programmiert wurde. Die Applikation erlaubt eine direkte Kommunikation zwischen Benutzern, ohne dass ein zentraler Server benötigt wird.
Ziel des Projekts war es, die Kommunikation zwischen Benutzern zu ermöglichen, die ohne zentrale Server auskommt. Die Anwendung nutzt libp2p für die P2P-Kommunikation und mDNS für die Peer-Erkennung.

Die wesentlichen Ziele des Projekts waren:
\begin{itemize}
    \item Entwicklung einer einfachen Chat-Anwedung
    \item P2P Kommunikation ohne zentralen Server
    \item Auswahl geeigneter Technologien und Methoden zur Ermöglichung der Kommunikation zwischen Peers
    \item Sicherstellung der Benutzerfreundlichkeit der Anwendung
    \item Automatische Erkennung von Peers und Verbindungsaufbau mit ihnen
    \item Bedienung über die Kommandozeile
\end{itemize}

Unsere Ergebnisse zeigen, dass die Verwendung von Go, libp2p und mDNS eine effektive Lösung für die Entwicklung einer P2P-Chat-Anwendung darstellt. Die Implementierung einer Benutzeroberfläche mit bubblecharm erlaubt eine einfache Bedienung der Anwendung.

Im Vergleich zu anderen P2P-Chat-Anwendungen, die oft komplexere Protokolle und Technologien verwenden, haben wir uns bewusst für eine einfachere und leicht verständliche Lösung entschieden. Dies ermöglicht es auch weniger erfahrenen Entwicklern, die Anwendung nachzuvollziehen und weiterzuentwickeln.

\newpage