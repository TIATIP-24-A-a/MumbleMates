\section{Hauptteil}

In diesem Abschnitt wird beschrieben, wie die formulierten Ziele erreicht wurden. Es wird beschrieben, welche Werkzeuge, Bauteile und Methoden eingesetzt wurden, um die Ziele zu erreichen und welche grundlegenden Probleme zu überwinden waren. Jede Person, die den Bericht liest, sollte dadurch in der Lage sein, die ausgeführten Arbeiten zu wiederholen und die Ergebnisse zu reproduzieren.

\subsection{Iteration 1: Grundlegende Architektur und Kommunikation}

\subsubsection{Problemstellung}
Die erste Herausforderung bestand darin, eine grundlegende Peer-to-Peer (P2P) Kommunikationsarchitektur zu entwickeln, die es den Benutzern ermöglicht, Nachrichten direkt miteinander auszutauschen, ohne einen zentralen Server zu verwenden.

\subsubsection{Werkzeuge und Methoden}
Für die Implementierung der P2P-Kommunikation haben wir die folgenden Werkzeuge und Technologien verwendet:
\begin{itemize}
    \item \textbf{Programmiersprache:} Go
    \item \textbf{Bibliotheken:} libp2p, mDNS
    \item \textbf{Entwicklungsumgebung:} Visual Studio Code
\end{itemize}

\subsubsection{Ergebnisse}
Wir haben eine einfache P2P-Kommunikationsarchitektur entwickelt, bei der jeder Benutzer als Peer fungiert. Die Kommunikation erfolgt über libp2p und mDNS für die Peer-Erkennung. Die grundlegende Funktionalität wurde erfolgreich implementiert und getestet.

\subsection{Iteration 2: Benutzeroberfläche und Benutzerfreundlichkeit}

\subsubsection{Problemstellung}
Die nächste Herausforderung bestand darin, eine benutzerfreundliche Oberfläche zu entwickeln, die es den Benutzern ermöglicht, einfach und intuitiv Nachrichten zu senden und zu empfangen.

\subsubsection{Werkzeuge und Methoden}
Für die Entwicklung der Benutzeroberfläche haben wir die folgenden Werkzeuge und Technologien verwendet:
\begin{itemize}
    \item \textbf{Programmiersprache:} Go
    \item \textbf{Bibliotheken:} Fyne (für GUI-Entwicklung)
    \item \textbf{Design-Tools:} Figma für die Erstellung von Mockups
\end{itemize}

\subsubsection{Ergebnisse}
Wir haben eine grafische Benutzeroberfläche (GUI) entwickelt, die es den Benutzern ermöglicht, Nachrichten zu senden und zu empfangen. Die GUI wurde basierend auf Benutzerfeedback iterativ verbessert, um die Benutzerfreundlichkeit zu erhöhen.

\subsection{Iteration 3: Sicherheit und Datenschutz}

\subsubsection{Problemstellung}
Die letzte Herausforderung bestand darin, die Sicherheit und den Datenschutz der Kommunikation zu gewährleisten, um sicherzustellen, dass Nachrichten nicht von unbefugten Dritten abgefangen oder gelesen werden können.

\subsubsection{Werkzeuge und Methoden}
Für die Implementierung der Sicherheitsmaßnahmen haben wir die folgenden Werkzeuge und Technologien verwendet:
\begin{itemize}
    \item \textbf{Programmiersprache:} Go
    \item \textbf{Bibliotheken:} libp2p-crypto
    \item \textbf{Methoden:} End-to-End-Verschlüsselung (E2EE)
\end{itemize}

\subsubsection{Ergebnisse}
Wir haben eine End-to-End-Verschlüsselung implementiert, die sicherstellt, dass Nachrichten nur von den beabsichtigten Empfängern gelesen werden können. Die Verschlüsselung wurde erfolgreich getestet und gewährleistet die Vertraulichkeit der Kommunikation.

\subsection{Zusammenfassung}
Durch die iterative Entwicklung und den Einsatz geeigneter Werkzeuge und Methoden konnten wir die formulierten Ziele erreichen. Die entwickelte P2P-Chat-Anwendung ermöglicht eine sichere und benutzerfreundliche Kommunikation zwischen den Benutzern. Die beschriebenen Schritte und verwendeten Technologien ermöglichen es anderen, die ausgeführten Arbeiten zu wiederholen und die Ergebnisse zu reproduzieren.