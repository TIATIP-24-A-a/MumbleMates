\section{Systemarchitektur}

\subsection{Dezentrale Architektur}
Der P2P-Chat basiert auf einer dezentralen Architektur, bei der jeder Knotenpunkt im Netzwerk als Peer agiert. Im Gegensatz zu zentralen Servern gibt es bei diesem System keine zentrale Instanz, die die Kommunikation verwaltet. Jeder Peer ist direkt mit anderen Peers verbunden, was für erhöhte Sicherheit und Datenschutz sorgt.

\subsection{Kommunikation zwischen Peers}
Die Kommunikation zwischen Peers erfolgt direkt über TCP/IP-Sockets. Jeder Peer hat eine einzigartige IP-Adresse und Portnummer, über die er Verbindungen zu anderen Peers herstellen kann.

\section{Go Implementierung}

\subsection{Go als Programmiersprache}

Go (oder Golang) wurde gewählt, da es für die Entwicklung von Netzwerksoftware besonders gut geeignet ist. Es bietet einfache Unterstützung für parallele Verarbeitung (Concurrency) durch Goroutinen und Kanäle, was bei der Entwicklung eines Echtzeit-Chat-Systems von großem Vorteil ist.

\subsection {Go-Tools und Libraries}

Für die Implementierung des P2P-Chats in Go wurden folgende Werkzeuge und Bibliotheken verwendet:

\begin{itemize}
    \item net: Zum Erstellen und Verwenden von TCP- und UDP-Verbindungen.
    \item libp2p: Eine Library für die Entwicklung von dezentralen Anwendungen, die auf dem Peer-to-Peer-Prinzip basieren.
\end{itemize}

\subsection{Architektur der Implementierung}

Die Architektur des Go-Programms basiert auf einem client-serverähnlichen Modell, wobei jedoch jeder Peer sowohl Server als auch Client ist. Das System ist so aufgebaut, dass jeder Peer beim Starten seine eigene ID (z.B. IP-Adresse) generiert und mit anderen Peers kommuniziert.

\begin{itemize}
    \item Server-Komponente: Jeder Peer hört auf einem bestimmten Port und akzeptiert eingehende Verbindungen von anderen Peers.
    \item Client-Komponente: Jeder Peer kann über Anfragen an andere Peers senden, um Nachrichten zu übermitteln.
\end{itemize}


\subsection{Nachrichteunübertragung}

\section{Peer-Discovery und -Verwaltung}

\subsection{Peer-Discovery}

Für die Peer-Discovery wird ein Bootstrap-Server verwendet. Zu Beginn verbindet sich jeder Peer mit dem Bootstrap-Server, um eine Liste anderer Peers zu erhalten. Dies ermöglicht es, neue Peers im Netzwerk zu finden. Sobald ein Peer die IP-Adressen anderer Peers kennt, kann er eine direkte Verbindung zu diesen herstellen.

\subsection{Peer-Verwaltung}

Das Netzwerk von Peers ist dynamisch. 
Ein Peer kann jederzeit das Netzwerk betreten oder verlassen. Beim Verlassen eines Peers wird der verbleibenden Peers die neue Konfiguration (Liste der aktiven Peers) übermittelt, sodass das Netzwerk stets aktuell bleibt.

\section{Fehlerbehandlung und Robustheit}

Wenn eine Verbindung zu einem Peer unterbrochen wird, versucht der Peer automatisch, die Verbindung nach einer festgelegten Zeitspanne erneut aufzubauen. Dieser Prozess hilft dabei, die Stabilität des Netzwerks zu gewährleisten.

\section{Performance und Skalierbarkeit}

\subsection{Lastverteilung}

Da es keine zentrale Instanz gibt, die alle Verbindungen verwaltet, wird die Last gleichmäßig auf alle Peers im Netzwerk verteilt. Jeder Peer ist für die Verwaltung seiner eigenen Verbindungen verantwortlich, wodurch die Skalierbarkeit des Systems erhöht wird.

\newpage
