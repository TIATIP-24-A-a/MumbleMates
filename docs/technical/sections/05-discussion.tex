\section{Diskussion}

In diesem Kapitel werden die einleitend gestellten Fragen beantwortet und die Beziehungen zwischen den eigenen und fremden Ergebnissen hergestellt. Die in der Einleitung formulierten Fragen und Ziele werden hier diskutiert und in einen breiteren Kontext gestellt. Es erfolgt ein Vergleich mit anderen Arbeiten, und weniger objektive Interpretationen sowie eigene Meinungen, Thesen und Selbstreflexion werden aufgeführt.

\subsection{Beantwortung der einleitenden Fragen und Ziele}

In der Einleitung wurden folgende Fragen und Ziele formuliert:
\begin{itemize}
    \item Wie kann eine einfache und sichere Peer-to-Peer (P2P) Chat-Anwendung entwickelt werden?
    \item Welche Technologien und Methoden sind am besten geeignet, um die Kommunikation zwischen Peers zu ermöglichen?
    \item Wie kann die Benutzerfreundlichkeit der Anwendung sichergestellt werden?
    \item Welche Sicherheitsmassnahmen sind notwendig, um die Vertraulichkeit der Kommunikation zu gewährleisten?
\end{itemize}

Unsere Ergebnisse zeigen, dass die Verwendung von Go, libp2p und mDNS eine effektive Lösung für die Entwicklung einer P2P-Chat-Anwendung darstellt. Die Implementierung einer grafischen Benutzeroberfläche mit bubbletea hat die Benutzerfreundlichkeit erheblich verbessert.

\subsection{Vergleich mit anderen Arbeiten}

Im Vergleich zu anderen P2P-Chat-Anwendungen, die oft komplexere Protokolle und Technologien verwenden, haben wir uns bewusst für eine einfachere und leicht verständliche Lösung entschieden. Dies ermöglicht es auch weniger erfahrenen Entwicklern, die Anwendung nachzuvollziehen und weiterzuentwickeln.

\subsection{Eigene Meinungen und Thesen}

Wir sind der Meinung, dass die Wahl von Go und libp2p eine ausgezeichnete Entscheidung war, da diese Technologien eine robuste und skalierbare P2P-Kommunikation ermöglichen. Die Verwendung von mDNS für die Peer-Erkennung hat sich als sehr effektiv erwiesen und könnte auch in anderen Projekten von großem Nutzen sein.

\subsection{Selbstreflexion}

Rückblickend hätten wir möglicherweise mehr Zeit in die Planung und Untersuchung alternativer libraries investieren können.

\subsection{Schlussfolgerungen}

Die Schlussfolgerungen aus diesem Bericht sind klar: Eine einfache, aber effektive P2P-Chat-Anwendung kann mit den richtigen Werkzeugen und Methoden entwickelt werden. Die iterative Entwicklung für den Erfolg einer solchen Anwendung. Die Ergebnisse unserer Arbeit zeigen, dass es möglich ist, eine einfache und benutzerfreundliche P2P-Chat-Anwendung zu entwickeln, die den Anforderungen der Benutzer gerecht wird.

Alle in der Diskussion aufgeführten Punkte wurden in den vorgängigen Kapiteln behandelt und basieren auf den dort beschriebenen Ergebnissen und Erfahrungen.
\newpage
